% % Mengubah keterangan `Abstract` ke bahasa indonesia.
% % Hapus bagian ini untuk mengembalikan ke format awal.
% % \renewcommand\abstractname{Abstrak}

% \begin{abstract}

%   % Ubah paragraf berikut sesuai dengan abstrak dari penelitian.
%   \lipsum[1]


% \end{abstract}

% % Mengubah keterangan `Index terms` ke bahasa indonesia.
% % Hapus bagian ini untuk mengembalikan ke format awal.
% % \renewcommand\IEEEkeywordsname{Kata kunci}

% \begin{IEEEkeywords}

%   % Ubah kata-kata berikut sesuai dengan kata kunci dari penelitian.
%   Deep Learning

% \end{IEEEkeywords}

% Change the `Abstract` label to Indonesian.
% Remove this section to revert to the original format.
\renewcommand\abstractname{Abstract}

\begin{abstract}

  % Update the following paragraph according to the research abstract.
The deaf use sign language as their primary means of communication. According to GERKATIN, there are at least 2.9 million deaf people. This is not matched by the general public's knowledge of sign language, resulting in communication difficulties between the deaf and those around them, limiting the improvement of their quality of life. Current translator systems are still limited to word-level translation only and have not made efforts to create inclusive systems. This research developed a BISINDO translation system using the LSTM architecture. The system has been implemented on Intel NUC with the ability to translate sign movements in real-time. Users can form commonly used sentences daily and convert them into voice media with the help of control sign movements. Based on the tests conducted, the system can adapt to differences in light intensity, distance, and subjects other than the author, with the highest accuracy reaching 100\%. This system can be a solution to overcome communication barriers between the deaf and the general public.

\end{abstract}

% Change the `Index terms` label to Indonesian.
% Remove this section to revert to the original format.
\renewcommand\IEEEkeywordsname{Index terms}

\begin{IEEEkeywords}

  % Update the following keywords according to the research keywords.
  Deaf, BISINDO, LSTM, Intel NUC

\end{IEEEkeywords}
