% Mengubah keterangan `Abstract` ke bahasa indonesia.
% Hapus bagian ini untuk mengembalikan ke format awal.
\renewcommand\abstractname{Abstrak}

\begin{abstract}

  % Ubah paragraf berikut sesuai dengan abstrak dari penelitian.
Tunarungu menggunakan bahasa isyarat sebagai bahasa komunikasi utama. Menurut GERKATIN terdapat setidaknya 2,9 juta orang penyandang tunarungu. Hal ini tidak diikuti dengan pengetahuan masyarakat umum mengenai bahasa isyarat yang berdampak pada sulitnya komunikasi tunarungu dengan masyarakat sekitar sehingga adanya keterbatasan dalam peningkatan kualitas hidup mereka. Sistem penerjemah saat ini masih terbatas dalam menerjemahkan dalam bentuk kata saja dan belum adanya upaya dalam membuat sistem yang bersifat inklusif. Pada penelitian ini telah dikembangkan sistem penerjemah BISINDO menggunakan arsitektur LSTM. Sistem telah diimplementasikan pada Intel NUC dengan kemampuan dalam menerjemahkan gerakan isyarat secara \emph{real time}. Pengguna dapat membentuk kalimat - kalimat yang umum digunakan sehari - hari dan mengkonversinya ke media suara dengan bantuan gerakan isyarat kontrol. Berdasarkan pengujian yang telah dilakukan, didapat bahwa sistem dapat beradaptasi dengan adanya perbedaan intensitas cahaya, jarak, serta subjek yang berbeda dengan penulis dengan akurasi tertinggi mencapai 100\%. Sistem ini dapat menjadi solusi dalam mengatasi hambatan komunikasi antara tunarungu dengan khalayak umum.

\end{abstract}

% Mengubah keterangan `Index terms` ke bahasa indonesia.
% Hapus bagian ini untuk mengembalikan ke format awal.
\renewcommand\IEEEkeywordsname{Kata kunci}

\begin{IEEEkeywords}

  % Ubah kata-kata berikut sesuai dengan kata kunci dari penelitian.
  Tunarungu, BISINDO, LSTM, Intel NUC

\end{IEEEkeywords}
