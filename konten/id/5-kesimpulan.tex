% Ubah judul dan label berikut sesuai dengan yang diinginkan.
\section{Kesimpulan}
\label{sec:kesimpulan}

% Ubah paragraf-paragraf pada bagian ini sesuai dengan yang diinginkan.

Berdasarkan penelitian yang telah dilaksanakan dapat diambil kesimpulan sebagai berikut:

\begin{enumerate}[nolistsep]
    \item Sistem penerjemah bahasa isyarat telah berhasil diimplementasikan pada Intel \emph{Next Unit Computing} (NUC) dan dapat berjalan secara \emph{real time}. 
    \item Model LSTM memiliki performa paling baik dengan penggunaan \emph{layer TimeDistributed} yang diikuti dengan 2 \emph{layer} LSTM pada akurasi 0.99 atau 99\%.
    \item Intensitas cahaya 125 lux atau kondisi ruangan terang menghasilkan klasifikasi terbaik dengan akurasi sebesar 100\%.
    \item Jarak kamera terhadap pengguna sebesar 300 cm menghasilkan klasifikasi terbaik dengan akurasi sebesar sebesar 100\%.
    \item Model berhasil beradaptasi dengan subjek selain penulis dengan akurasi sebesar 92.5\% untuk subjek perempeuan dan laki - laki.
    \item Sistem penerjemah dapat membentuk kalimat dan mengkonversi menjadi suara dengan tingkat keberhasilan sebesar 85.7\%.
  
\end{enumerate}

Berdasarkan penelitian yang telah dilakukan, adapun saran yang dapat dipertimbangkan untuk pengembangan lebih lanjut adalah sebagai berikut:

\begin{enumerate}[nolistsep]

  \item Menambahkan variasi jarak, intensitas cahaya, dan subjek berbeda untuk menguji bagaim\\ana kemampuan model dalam beradaptasi dengan serangkaian perubahan yang terjadi.
  \item Mempertimbangkan menggunakan metode lain, seperti CNN-LSTM untuk dapat melakukan ekstraksi \emph{feature} dalam bentuk citra sehingga dapat melakukan klasifikasi gerakan bahasa isyarat dengan lebih baik dan akurat lagi.

\end{enumerate}