% Ubah judul dan label berikut sesuai dengan yang diinginkan.
\section{Pendahuluan}
\label{sec:pendahuluan}

% Ubah paragraf-paragraf pada bagian ini sesuai dengan yang diinginkan.

Bahasa isyarat merupakan bahasa yang mengutamakan komunikasi manual dengan mengkombinasikan bentuk tangan, orientasi gerak tangan, lengan, bibir, ataupun ekspresi mimik wajah untuk mengungkapkan sesuatu. Tunarungu merupakan kondisi ketidakmampuan seorang dalam menangkap rangsangan secara auditori melalui indra pendengarannya \cite{maulida2017}. Penyandang tunarungu menggunakan bahasa isyarat dalam berkomunikasi, baik kepada sesama penyandang tunarungu ataupun masyarakat sekitar. Di Indonesia, bahasa isyarat Indonesia (BISINDO) merupakan bahasa isyarat yang lebih umum digunakan dalam kehidupan sehari - hari. GERKATIN (Gerakan untuk Kesejahteraan Tunarungu Indonesia) mencatat terdapat setidaknya 2,9 juta orang atau sekitar 1,25\% dari total populasi penduduk Indonesia yang merupakan masyarakat penyandang tunarungu \cite{evitasari2015}. Namun, kurangnya pemahaman masyarakat tentang bahasa isyarat menyebabkan penyandang tunarungu kesulitan berkomunikasi, sehingga membatasi kualitas hidup mereka. Perkembangan teknologi saat ini telah menghasilkan berbagai inovasi yang membantu manusia dalam kehidupan sehari-hari, termasuk \emph{deep learning}.Salah satu arsitekturnya \emph{deep learning}, \emph{Long Short-Term Memory} (LSTM), dapat menyimpan informasi masa lampau dan mempelajari berbentuk data sekuensial. LSTM ideal untuk menerjemahkan bahasa isyarat karena memiliki pola sekuensial \cite{sadli2020}. Penelitian sebelumnya oleh Putri et al. berhasil mengembangkan deteksi real-time BISINDO dengan akurasi 65\% pada 30 kelas kata \cite{putri2022}. Sedangkan penelitian oleh Suhartijo dan Aljabar mencapai 86\% akurasi pada 10 kelas menggunakan LSTM \cite{aljabar2020}. Selain itu, perkembangan teknologi juga melahirkan perangkat bernama \emph{mini computer} yangmemiliki fungsi seperti komputer biasa, namun dalam ukuran lebih kecil dengan keunggulan ringan, ringkas, efisiensi energi, dan harga terjangkau. Pada tahun 2013, Intel mengeluarkan mini PC bernama Intel Next Unit Computing (NUC), yang menjadi solusi perangkat dengan kemampuan komputasi dan mobilitas tinggi \cite{minny2023}. Oleh karena itu, dibutuhkan suatu sistem penerjemah BISINDO ke media suara sebagai salah satu solusi dalam mengatasi masalah komunikasi antar penyandang tuna rungu dengan masyarakat umum dan mendukung GERKATIN dalam menggantikan bahasa isyarat nasional. Sistem penerjemah saat ini masih diimplementasikan secara pada laptop saja sehingga Intel \emph{Next Unit Computing} menjadi perangkat yang akan diimplementasikan sistem penerjemah bahasa isyarat sebagai upaya awal dalam menciptakan sistem penerjemah yang dapat dijangkau oleh masyarakat luas.