% Ubah judul dan label berikut sesuai dengan yang diinginkan.
\section{Introduction}
\label{sec:introduction}

% Ubah paragraf-paragraf pada bagian ini sesuai dengan yang diinginkan.

% Pesatnya perkembangan roket yang merupakan \lipsum[2-4]

% Pembahasan pada paper ini dimulai dengan presentasi mengenai penelitian lain (Bagian \ref{sec:penelitianterkait}).
% Kemudian dilanjutkan dengan penjelasan mengenai arsitektur dari sistem yang dibuat (Bagian \ref{sec:arsitektur}).
% Berdasarkan hal tersebut, kami menunjukkan lorem ipsum (Bagian \ref{sec:loremipsum}).
% Terakhir, didapatkan kesimpulan dari penelitian yang telah dilakukan (Bagian \ref{sec:kesimpulan}).

Sign language is a language that prioritizes manual communication by combining hand shapes, hand movement orientations, arms, lips, or facial expressions to express something. Deafness is a condition where a person cannot receive auditory stimuli through their hearing senses \cite{maulida2017}. Deaf individuals use sign language to communicate, both with other deaf individuals and the surrounding community. In Indonesia, Indonesian Sign Language (BISINDO) is more commonly used in daily life. GERKATIN notes that at least 2.9 million people, or around 1.25\% of Indonesia's total population, are deaf \cite{evitasari2015}. However, the lack of public understanding of sign language makes it difficult for deaf people to communicate, thereby limiting their quality of life.

Technological advancements today have led to various innovations that help humans in daily life, including \emph{deep learning}. One of its architectures, \emph{Long Short-Term Memory} (LSTM), can store past information and learn sequential data patterns. LSTM is ideal for translating sign language due to its sequential pattern \cite{sadli2020}. Previous research by Putri et al. successfully developed a real-time detection system for Indonesian Sign Language (BISINDO) with 65\% accuracy in 30 word classes \cite{putri2022}, while research by Suhartijo and Aljabar achieved 86\% accuracy in 10 classes using LSTM \cite{aljabar2020}.

In addition, technological advancements have also given rise to devices called \emph{mini computers}, which function like regular computers but in a smaller size with the advantages of being lightweight, compact, energy-efficient, and affordable. In 2013, Intel released a mini PC called Intel \emph{Next Unit Computing} (NUC), which provides a solution for devices with high computing capabilities and mobility \cite{minny2023}. Therefore, a system that translates Indonesian Sign Language (BISINDO) into voice media is needed as a solution to overcome communication barriers between deaf people and the general public and support GERKATIN in replacing the national sign language. The current translation system is still only implemented on laptops, so the Intel \emph{Next Unit Computing} will be the device on which the Indonesian Sign Language translation system is implemented as an initial step in creating a translation system accessible to the general public.
