% Ubah judul dan label berikut sesuai dengan yang diinginkan.
\section{Conclusion}
\label{sec:conclusion}

% Ubah paragraf-paragraf pada bagian ini sesuai dengan yang diinginkan.

Based on the research conducted, the following conclusions can be drawn:

\begin{enumerate}[nolistsep]
    \item The sign language translator system has been successfully implemented on Intel \emph{Next Unit Computing} (NUC) and can operate in real-time.
    \item The LSTM model has the best performance using the \emph{TimeDistributed layer} followed by 2 LSTM layers, with an accuracy of 0.99 or 99\%.
    \item Light intensity of 125 lux or bright room conditions produces the best classification with an accuracy of 100\%.
    \item A camera distance of 300 cm from the user results in the best classification with an accuracy of 100\%.
    \item The model successfully adapts to subjects other than the author, with an accuracy of 92.5\% for both female and male subjects.
    \item The translation system can form sentences and convert them into voice with a success rate of 85.7\%.
\end{enumerate}

Based on the research conducted, the following suggestions can be considered for further development:

\begin{enumerate}[nolistsep]
  \item Adding variations in distance, light intensity, and different subjects to test how well the model can adapt to a series of changes.
  \item Consider using other methods, such as CNN-LSTM, to extract features in the form of images, allowing for better and more accurate classification of sign language movements.
\end{enumerate}

