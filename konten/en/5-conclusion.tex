% Ubah judul dan label berikut sesuai dengan yang diinginkan.
\section{Conclusion}
\label{sec:conclusion}

% Ubah paragraf-paragraf pada bagian ini sesuai dengan yang diinginkan.

Based on the research conducted, the following conclusions can be drawn:

\begin{enumerate}[nolistsep]
  \item The sign language translation system has been successfully implemented on Intel \emph{Next Unit Computing} (NUC) and can operate in \emph{real-time}.
  \item The LSTM model performed best with the use of the \emph{TimeDistributed layer} followed by 2 LSTM layers at an accuracy of 99\%.
  \item A light intensity of 125 lux or a bright room condition produced the best classification with an accuracy of 100\% and a relatively high average FPS value of 12.905.
  \item A camera distance of 300 cm from the user produced the best classification with an accuracy of 100\%, but with a relatively low average FPS value of 10.746.
  \item The model successfully adapted to subjects other than the author with an accuracy of 92.5\% for both female and male subjects, with a relatively high average FPS value of 11.702.
  \item The translation system can form sentences and convert them into speech with a success rate of 85.7\% and a relatively high average FPS value of 11.678.
\end{enumerate}

Based on the research conducted, the following suggestions can be considered for further development:

\begin{enumerate}[nolistsep]
  \item Adding variations in distance, light intensity, and different subjects to test how well the model can adapt to a series of changes.
  \item Consider using other methods, such as CNN-LSTM, to extract features in the form of images, allowing for better and more accurate classification of sign language movements.
\end{enumerate}

