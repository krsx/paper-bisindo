% nama
\newcommand{\name}{I Putu Krisna Erlangga}
\newcommand{\authorname}{Erlangga, I Putu Krisna}
\newcommand{\nickname}{Krisna}
\newcommand{\advisor}{Dr. Eko Mulyanto Yuniarno, S.T., M.T.}
\newcommand{\coadvisor}{Dr. Diah Puspito Wulandari, S.T., M.Sc.}

% identitas
\newcommand{\nrp}{5024 20 1055}
\newcommand{\advisornip}{19740907 200212 1 001}
\newcommand{\coadvisornip}{19680601 199512 1 009}
\newcommand{\email}{5024201055@student.its.ac.id}

\newcommand{\advisoremail}{ekomulyanto@ee.its.ac.id}
\newcommand{\coadvisoremail}{diah@te.its.ac.id}

% judul
\newcommand{\tatitle}{PENERJEMAH BAHASA ISYARAT INDONESIA (BISINDO) KE MEDIA SUARA MENGGUNAKAN \emph{LONG SHORT-TERM MEMORY} (LSTM) BERBASIS INTEL \emph{NEXT UNIT COMPUTING} (NUC)}
\newcommand{\engtatitle}{\emph{TRANSLATOR OF INDONESIAN SIGN LANGUAGE (BISINDO) TO VOICE MEDIA USING LONG SHORT-TERM MEMORY (LSTM) BASED ON INTEL NEXT UNIT COMPUTING (NUC)}}

% tempat
\newcommand{\place}{Surabaya}

% jurusan
\newcommand{\studyprogram}{Teknik Komputer}
\newcommand{\engstudyprogram}{Computer Engineering}

% fakultas
\newcommand{\faculty}{Teknologi Elektro dan Informatika Cerdas}
\newcommand{\engfaculty}{Intelligence Electrical and Informatics Technology}

% singkatan fakultas
\newcommand{\facultyshort}{FTEIC}
\newcommand{\engfacultyshort}{ELECTICS}

% departemen
\newcommand{\department}{Teknik Komputer}
\newcommand{\engdepartment}{Computer Engineering}
